\begin{thebibliography}{1}


\bibitem {lombardi}
T. E. Lombardi. The classification of style in fine-art painting. ETD Collection for Pace University, 2005.

\bibitem {jou}
J. Jou and S. Agrawal. Artist identification for renaissance paintings.

\bibitem {mensink}
T. Mensink and J. van Gemert. The rijksmuseum challenge: Museum-centered visual recognition. 2014

\bibitem {saleh}
B. Saleh and A. M. Elgammal. Large-scale classification of fine-art paintings: Learning the right metric on the right feature. CoRR, abs/1505.00855, 2015.

\bibitem {nitin}
Nitin Viswanathan, Artist Identification with Convolutional Neural Networks

\bibitem {ga}
Eyal Wirsansky, ”Hands on Genetic Algorithms with Python” book

\bibitem {perez}
Perez, Luis; Wang, Jason. The effectiveness of data augmentation in image classification using deep learning. arXiv preprint arXiv:1712.04621, 2017.

\bibitem {KINGMA}
KINGMA, Diederik P.; BA, Jimmy. Adam: A method for stochastic optimization. arXiv preprint arXiv:1412.6980, 2014.

\bibitem{lisha}
LI, Lisha, et al. Hyperband: A novel bandit-based approach to hyperparameter optimization. The Journal
of Machine Learning Research, 2017, 18.1: 6765-6816

\bibitem {SELVARAJU}
SELVARAJU, Ramprasaath R., et al. Grad-cam: Visual explanations from deep networks via gradient-based localization. In: Proceedings of the IEEE international conference on computer vision. 2017. p. 618-626.

\bibitem {szegedy}
Szegedy, et. al, Rethinking the Inception Architecture for Computer Vision

\end{thebibliography}
